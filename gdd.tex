\documentclass[22pt]{article}
\usepackage[utf8]{inputenc}
\usepackage{polski}

\title{\huge{Left On Site}}
\author{Aleksander Modzelewski}


\begin{document}
\maketitle

\begin{center}
Projekt zaliczeniowy\\
informatyka, 3 semestr
\end{center}
\newpage
\tableofcontents
\newpage

\section{Informacje ogólne}
\subsection{Tło}
Statek kosmiczny OSK-1 mający badać pewną planetę
podobną do Ziemi - Aftergaję rozbija się na niej. Podczas awaryjnego lądowania na planecie cała załoga,
 poza jedną osobą ginie. Osoba ta oczekuje na misję ratunkową próbując przeżyć.
\subsection{Filozofia i cel gry}
Gracz wciela się w rozbitka statku OSK-1. Jego celem jest przeżycie do czasu zbudowania przez niego sygnalizatora, który wyśle lokalizację gracza do stacji kosmicznej po czym na planetę zostanie zesłana kapsuła ratunkowa. Gracz może 
wejść do tej kapsuły kończąc grę lub dalej eksplorować planetę. Do czasu ukończenia sygnalizatora gracz buduje swoją bazę na planecie zdobywając jedzenie, surowce i craftując itemy. 
\section{Postać}
\subsection{interfejs gry}
W lewym górnym rogu znajduje się minimapa. Poniżej minimapy znajdziemy pasek
życia, głodu i ciężaru ekwipunku. W prawym górnym rogu znajdziemy zegar wraz z
licznikiem dni pod nim. W dolnej części ekranu znajduje się pasek szybkiego dostępu do przedmiotów. Żeby wybrać przedmiot(umieścić go w ręce) należy kliknąć go w pasku szybkiego dostępu. Po wciśnięciu 'q' otwiera nam się ekwipunek wraz z 
ograniczonym miejscem na crafting. Jeśli z posiadanych przedmiotów możemy zbudować przedmiot pojawi się on w po lewej jako opcja, którą można wybrać myszką. W tym miejscu możemy przenieść przedmioty do paska szybkiego dostępu. 
\subsection{sterowanie}
Sterowanie bohaterem odbywa się poprzez klawisze WASD. Jeśli w ręce mamy przedmiot którym można zaatakować możemy przytrzymać spację by to zrobić. Im bardziej zapełniony wskaźnik ciężaru ekwipunku tym wolniej porusza się bohater. Wchodzenie w interakcję z elementami świata odbywa się poprzez kliknięcie na element myszką, a następnie wybranie opcji z okienka wyboru.

\section{Świat gry}
\subsection{surowce}
-- sucha trawa\\
-- patyki\\
-- kamienie\\
-- krzem\\
-- żelazo\\
-- deski\\
-- siarka\\
-- węgiel\\
-- guma\\
-- ortogon\\
-- tranton\\
-- omnoton\\
-- skóra\\
-- łuska wężopancernika\\
\subsection{craftowalne itemy}
-- siekiera(kamienna i żelazna)\\
-- łopata(kamienna i żelazna)\\
-- kilof(kamienny i żelazny)\\
-- ognisko\\
-- wędka\\
-- pochodnia\\
-- pistolet\\
-- proch\\
-- naboje\\
-- pochodnia\\
-- latarka\\
-- stół do craftowania\\
-- tranzystor\\
-- ogniwo ortogonowe\\
-- ogniwo trantonowe\\
-- ogniwo omnotonowe\\
-- latarnia\\
-- akumulator\\
-- baterie słoneczne\\
-- namiot\\
-- skrzynia\\
-- gwoździe\\
-- kuchenka na węgiel\\
-- kuchenka elektryczna\\
-- działko ortogonowe\\
-- pancerz\\
-- sygnalizator\\
\subsection{zwierzęta neutralne}
-- szopokrab\\
-- ważkoliber\\
-- ryborurka\\
\subsection{zwierzęta agresywne}
-- nosacz\\
-- wilkowół\\
-- wężopancernik\\
\subsection{jedzenie}
-- mięso\\
-- złowiona ryborurka\\
-- owoce\\
-- nektar ważkolibra\\
-- jajka wężopancernika\\

\section{Udźwiękowienie i grafika}
\subsection{Udźwiękowienie}
Muzyka zostanie wykonana w open-sourcowym programie LMMS. Do realizacji części 
dźwięków otoczenia zostaną wykorzystane assety z zasobów opengameart.org
\subsection{Grafika}
Grafika zostanie wykonana w stylu pixel-art.
Graficzne elementy charakterystyczne dla gry zostaną wykonane w programie
aseprite. Część elementów pochodzić będzie z zasobów opengameart.org.
\end{document}






